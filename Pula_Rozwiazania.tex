\documentclass{article}
\usepackage{polski}
\usepackage{amssymb}
\usepackage{amsmath}
\usepackage{amsthm}
\usepackage{graphicx}
\usepackage[utf8]{inputenc}

\newtheorem{problem}{Zadanie}
\newtheorem{lemma}{lemat}
\newtheorem{sublemma}{Lemma}[lemma]

\begin{document}

\title{Zbiór rozwiązań zadań z Puli Analizy Matematycznej}
\author{Alicja Kluczek, Marek Sommer}

\date{\today}
\maketitle
\newpage

\section{Liczby rzeczywiste. Kresy zbiorów.}

\begin{problem}
    Udowodnić, że liczba $\sqrt{7 + \sqrt{3}}$ jest
    niewymierna.
\end{problem}
\begin{proof}

    \begin{lemma}
        Liczba postaci $\sqrt{a + b}$ dla $a \in \mathbb{Q} \wedge b \notin \mathbb{Q}$
        nie jest liczbą wymierną.
    \end{lemma}
    \begin{proof}
        Załóżmy, że jest wymierne i równe $p$.
        Przeprowadzam ciąg przekształceń:
        $$ p^2 = a + b $$
        $$ p^2 - a = q, q \in \mathbb{Q} $$
        $$ q = b $$
        Co kończy dowód, ponieważ jest to równość nieprawdziwa.
        Zatem nie istnieje takie wymierne $p$.
    \end{proof}
    Aplikujemy lemat dla $a = 7$ i $b = \sqrt{3}$.
\end{proof}

\begin{problem}
    Rozstrzygnąć, czy liczba $\sqrt{\sqrt{5} + 3} + \sqrt{\sqrt{5} - 2}$
    jest wymierna.
\end{problem}
\begin{proof}
    Załóżmy, że jest to liczba wymierna $p$.
    Wiadomo, że $2 \sqrt{\sqrt{5} + 3}$ nie jest liczbą wymierną (lemat z zadania poprzedniego).
    $$
    \left\{ \begin{array}{ll}
        \sqrt{\sqrt{5} + 3} + \sqrt{\sqrt{5} - 2} = p\\
        (\sqrt{\sqrt{5} + 3} + \sqrt{\sqrt{5} - 2})(\sqrt{\sqrt{5} + 3}
         - \sqrt{\sqrt{5} - 2}) = 5\\
        (\sqrt{\sqrt{5} + 3} + \sqrt{\sqrt{5} - 2}) -
        (\sqrt{\sqrt{5} + 3} - \sqrt{\sqrt{5} - 2}) = 2 \sqrt{\sqrt{5} + 3}
    \end{array} \right.
    $$

    $$
    \left\{ \begin{array}{ll}
        \sqrt{\sqrt{5} + 3} + \sqrt{\sqrt{5} - 2} = p\\
        \sqrt{\sqrt{5} + 3} - \sqrt{\sqrt{5} - 2} = \frac{5}{p}\\
        (\sqrt{\sqrt{5} + 3} + \sqrt{\sqrt{5} - 2}) +
        (\sqrt{\sqrt{5} + 3} - \sqrt{\sqrt{5} - 2}) = 2 \sqrt{\sqrt{5} + 3}
    \end{array} \right.
    $$

    $$
    \left\{ \begin{array}{ll}
        \sqrt{\sqrt{5} + 3} + \sqrt{\sqrt{5} - 2} = p\\
        \sqrt{\sqrt{5} + 3} - \sqrt{\sqrt{5} - 2} = \frac{5}{p}\\
        p + \frac{5}{p} = 2 \sqrt{\sqrt{5} + 3}
    \end{array} \right.
    $$

    Sprzeczność, ponieważ lewa strona równania wymierna, a druga nie.
\end{proof}

\begin{problem}
    Niech $A \subset \mathbb{R}$ będzie zbiorem ograniczonym i
    $\lambda \in \mathbb{R}$. Zbiór $\lambda A$ określamy wzorem
    $$ \lambda A := \{ \lambda a : a \in A \} $$
    Oznaczmy $\sup A = M$ i $\inf A = m$. Wyznaczyć kresy zbioru
    $\lambda A$.
\end{problem}
\begin{proof}
    Załóżmy, że $\lambda > 0$.
    Twierdzę, że $\sup \lambda A = \lambda \sup A$.
    Z aksjomatów liczb rzeczywistych
    $$ a > b \,\wedge\, c > 0 \Rightarrow ac > bc $$
    Zatem:
    $$ \forall_{a \in A} M \geq a \Rightarrow \lambda M \geq \lambda a$$
    Zatem $\lambda M$ jest ograniczeniem górnym zbioru $\lambda A$.
    Załóżmy, że istnieje mniejsze ograniczenie $M'$.
    $$ \forall_{a \in A} M' >= \lambda a \,\wedge\, M' < \lambda M $$
    Wiadomo, że $\frac{1}{\lambda} > 0$, więc kolejny raz aplikując aksjomat:
    $$ \forall_{a \in A} \frac{1}{\lambda}M' >= a \,\wedge\, \frac{1}{\lambda}M' < M $$
    To by świadczyło o istnieniu mniejszego ograniczenia górnego zbioru $A$ niż
    $M$, sprzeczność. Analogicznie pokazujemy, że $\inf \lambda A = \lambda \inf A$.

    W przypadku, gdy $\lambda = 0$, $\lambda A $ jest zbiorem pustym lub
    singletonem $0$, więc zarówno infimum jak i supremum jest znane.

    Gdy $\lambda < 0$, dowód jest analogiczny jak przy $\lambda > 0$.
    Należy jednak pamiętać, że mnożenie stronami przez liczbę ujemną
    zmienia znak nierówności -- jednak nie jest to na tyle podobny dowód, że
    nie warto go prezentować.
\end{proof}

\begin {problem}
Wyznaczyć kresy zbioru
$$ \left\{ \frac{(-1)^n - m}{n + m} : \, n,m \in \mathbb{N} \right\} $$
\end{problem}
\begin{proof}
Dzielę problem na dwie części:\\
\subsection{Infimum}
Pokażę, że infimum jest równe $-1$.
$$ \frac{(-1)^n - m}{n + m} \geq -1$$
$$ (-1)^n - m \geq -n - m$$
$$ (-1)^n \geq -n $$
Co dla $n \in \mathbb{N}$ jest prawdą.
Dla $n = 1$ i $m = 1$ ułamek ma postać $ \frac{-1 - 1}{1 + 1} $ czyli $-1$.
Skoro $-1$ jest ograniczeniem dolnym i należy do zbioru, to jest infimum.
\subsection{Supremum}
Pokażę, że supremum jest równe $0$.
$$ \frac{(-1)^n - m}{n + m} \leq 0$$
$$ (-1)^n - m \leq 0$$
$$ (-1)^n \leq m$$
Co dla $m \in \mathbb{N}$ jest prawdą.
Dla $m = 1$ i $n \to \infty$ wartość wyrażenia dąży do $0$.
Skoro $0$ jest ograniczeniem górnym i istnieje ciąg wyrazów zbioru, które
do niego dążą, to jest supremum
\end{proof}

\begin{problem}
Znaleźć kresy zbioru $ \left\{ \frac{2nm}{n^2 + m^2} : n,m \in \mathbb{N}\right\}$.
\end{problem}
\begin{proof}
\dots
\end{proof}


\end{document}
